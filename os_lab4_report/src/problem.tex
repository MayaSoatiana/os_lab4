\section{Условие}
Требуется создать динамические библиотеки, которые реализуют заданный функционал. Далее использовать данные библиотеки 2-мя способами:
1. Во время компиляции (на этапе «линковки»/linking)
2. Во время исполнения программы. Библиотеки загружаются в память с помощью интерфейса ОС для работы с динамическими библиотеками

В конечном итоге, в лабораторной работе необходимо получить следующие части:
\begin{itemize}
    \item Динамические библиотеки, реализующие контракты, которые заданы вариантом;
    \item Тестовая программа (программа №1), которая используют одну из библиотек, используя информацию полученные на этапе компиляции;
    \item Тестовая программа (программа №2), которая загружает библиотеки, используя только их относительные пути и контракты.
\end{itemize}

Пользовательский ввод для обоих программ должен быть организован следующим образом:
\begin{enumerate}
    \item Если пользователь вводит команду «0», то программа переключает одну реализацию контрактов на другую (необходимо только для программы №2);
    \item «1 arg1 arg2 … argN», где после «1» идут аргументы для первой функции, предусмотренной контрактами. После ввода команды происходит вызов первой функции, и на экране появляется результат её выполнения;
    \item «2 arg1 arg2 … argM», где после «2» идут аргументы для второй функции, предусмотренной контрактами. После ввода команды происходит вызов второй функции, и на экране появляется результат её выполнения.
\end{enumerate}

Контракты и реализации функций:
\begin{enumerate}
    \item \textbf{Функция PrimeCount}: Подсчёт количества простых чисел на отрезке [A, B] (A, B - натуральные)
    \begin{itemize}
        \item \textbf{Сигнатура}: \texttt{int PrimeCount(int A, int B)}
        \item \textbf{Реализация 1}: Наивный алгоритм. Проверить делимость текущего числа на все предыдущие числа.
        \item \textbf{Реализация 2}: Решето Эратосфена
    \end{itemize}
    
    \item \textbf{Функция Square}: Подсчет площади плоской геометрической фигуры по двум сторонам
    \begin{itemize}
        \item \textbf{Сигнатура}: \texttt{float Square(float A, float B)}
        \item \textbf{Реализация 1}: Фигура прямоугольник (площадь = A × B)
        \item \textbf{Реализация 2}: Фигура прямоугольный треугольник (площадь = (A × B) / 2)
    \end{itemize}
\end{enumerate}

{\bfseries Цель работы:}
Приобретение практических навыков в:
\begin{itemize}
    \item Создание динамических библиотек
    \item Создание программ, которые используют функции динамических библиотек
\end{itemize}

{\bfseries Задание:}
Разработать динамические библиотеки, реализующие заданные контракты, и две тестовые программы, которые используют эти библиотеки разными способами.

Динамические библиотеки должны:
\begin{itemize}
    \item Реализовывать контракты функций \texttt{PrimeCount} и \texttt{Square};
    \item Иметь две различные реализации каждой функции (библиотека 1 и библиотека 2);
    \item Быть кроссплатформенными (работать на Windows как DLL и на Linux как SO).
\end{itemize}

Программа №1 должна:
\begin{itemize}
    \item Использовать одну из библиотек, используя информацию полученную на этапе компиляции;
    \item Поддерживать пользовательский ввод команд:
    \begin{itemize}
        \item Команда «0» — вывод информации о текущей реализации;
        \item Команда «1 A B» — вызов функции \texttt{PrimeCount(A, B)};
        \item Команда «2 A B» — вызов функции \texttt{Square(A, B)};
    \end{itemize}
    \item Выводить результаты выполнения функций в консоль.
\end{itemize}

Программа №2 должна:
\begin{itemize}
    \item Загружать библиотеки во время исполнения программы, используя только их относительные пути и контракты;
    \item Поддерживать пользовательский ввод команд:
    \begin{itemize}
        \item Команда «0» — переключение между реализациями библиотек (библиотека 1, библиотека 2);
        \item Команда «1 A B» — вызов функции \texttt{PrimeCount(A, B)} из текущей библиотеки;
        \item Команда «2 A B» — вызов функции \texttt{Square(A, B)} из текущей библиотеки;
    \end{itemize}
    \item Динамически загружать и выгружать библиотеки по требованию пользователя;
    \item Выводить результаты выполнения функций в консоль.
\end{itemize}

Провести анализ двух типов использования библиотек: статической линковки во время компиляции и динамической загрузки во время выполнения.

{\bfseries Вариант:} 19


