\section{Метод решения}

Для решения задачи применена архитектура с двумя динамическими библиотеками и двумя тестовыми программами, использующими эти библиотеки разными способами. Первая программа использует статическую линковку библиотеки на этапе компиляции, вторая программа загружает библиотеки динамически во время выполнения с использованием интерфейса ОС.

\subsection{Основной алгоритм работы}
\begin{enumerate}
    \item \textbf{Создание контрактов:} Разработаны заголовочные файлы (prime.h, square.h), определяющие интерфейсы функций, которые должны реализовывать динамические библиотеки
    \item \textbf{Реализация библиотек:} Созданы две динамические библиотеки с разными реализациями функций:
    \begin{itemize}
        \item Библиотека 1: наивный алгоритм подсчета простых чисел и площадь прямоугольника
        \item Библиотека 2: алгоритм "Решето Эратосфена" и площадь прямоугольного треугольника
    \end{itemize}
    \item \textbf{Статическое использование (программа №1):} Библиотека линкуется на этапе компиляции, функции вызываются напрямую
    \item \textbf{Динамическое использование (программа №2):} Библиотека загружается во время выполнения через Windows API, функции вызываются через указатели
    \item \textbf{Обработка команд пользователя:} Обе программы реализуют единый интерфейс ввода команд с обработкой ошибок
\end{enumerate}

\subsection{Особенности реализации}
Для обеспечения кроссплатформенности использованы условные директивы компиляции. Реализована корректная работа с динамическими библиотеками через механизмы экспорта/импорта функций. Обеспечена возможность переключения между реализациями в программе №2 без перекомпиляции.

\section{Описание программы}

Программа реализована в модульном стиле и состоит из шести основных компонентов.

\subsection{Модуль prime.h}
Определяет контракт для функции подсчета простых чисел:
\begin{itemize}
    \item \textbf{Структура:} Заголовочный файл с макросами для экспорта/импорта функций
    \item \textbf{Функция:} \texttt{int PrimeCount(int A, int B)} - подсчет простых чисел на отрезке [A, B]
    \item \textbf{Особенности:} Использование \texttt{extern "C"} для предотвращения декорирования имен, условная компиляция для Windows
\end{itemize}

\subsection{Модуль square.h}
Определяет контракт для функции вычисления площади:
\begin{itemize}
    \item \textbf{Структура:} Заголовочный файл с макросами для экспорта/импорта функций
    \item \textbf{Функция:} \texttt{float Square(float A, float B)} - вычисление площади фигуры
    \item \textbf{Особенности:} Проверка корректности входных данных (отрицательные значения)
\end{itemize}

\subsection{Модуль lib\_1 (первая реализация)}
Содержит первую реализацию контрактов:
\begin{itemize}
    \item \textbf{prime\_1.cpp:} Наивный алгоритм подсчета простых чисел (проверка делимости на все предыдущие числа)
    \item \textbf{square\_1.cpp:} Вычисление площади прямоугольника по формуле $S = A \times B$
    \item \textbf{Особенности:} Простая реализация, оптимальная для небольших диапазонов чисел
\end{itemize}

\subsection{Модуль lib\_2 (вторая реализация)}
Содержит вторую реализацию контрактов:
\begin{itemize}
    \item \textbf{prime\_2.cpp:} Алгоритм "Решето Эратосфена" для эффективного подсчета простых чисел
    \item \textbf{square\_2.cpp:} Вычисление площади прямоугольного треугольника по формуле $S = \frac{A \times B}{2}$
    \item \textbf{Особенности:} Использование вектора для оптимизации, эффективность для больших диапазонов
\end{itemize}

\subsection{Модуль program1/main.cpp (статическое использование)}
Реализует программу со статической линковкой:
\begin{itemize}
    \item \textbf{Архитектура:} Компиляция с библиотекой lib\_1, прямое использование функций
    \item \textbf{Интерфейс:} Консольное меню с командами 0, 1, 2, help, quit
    \item \textbf{Особенности:} Невозможность переключения реализаций без перекомпиляции
    \item \textbf{Обработка ввода:} Парсинг строк, валидация аргументов, обработка ошибок
\end{itemize}

\subsection{Модуль program2/main.cpp (динамическое использование)}
Реализует программу с динамической загрузкой:
\begin{itemize}
    \item \textbf{Класс DynamicLibrary:} Обертка для работы с Windows API загрузки библиотек
    \begin{itemize}
        \item \texttt{load()} - загрузка DLL файла
        \item \texttt{getFunction()} - получение указателя на функцию
        \item \texttt{unload()} - выгрузка библиотеки из памяти
    \end{itemize}
    \item \textbf{Интерфейс:} Консольное меню с возможностью переключения библиотек (команда 0)
    \item \textbf{Особенности:} Загрузка библиотек по требованию, использование указателей на функции
    \item \textbf{Обработка ошибок:} Проверка существования файлов, корректности загрузки, доступности функций
\end{itemize}

\subsection{Используемые системные вызовы и API}
\begin{itemize}
    \item \textbf{Windows API для динамических библиотек:} \texttt{LoadLibraryW}, \texttt{GetProcAddress}, \texttt{FreeLibrary}
    \item \textbf{Экспорт/импорт функций:} \texttt{\_\_declspec(dllexport)}, \texttt{\_\_declspec(dllimport)}
    \item \textbf{Обработка строк:} \texttt{std::stringstream}, \texttt{std::getline}, преобразование в широкие строки
    \item \textbf{Управление памятью:} \texttt{std::vector} для алгоритма "Решето Эратосфена"
\end{itemize}

Архитектура программы демонстрирует два принципиально разных подхода к использованию библиотек: статическую линковку на этапе компиляции и динамическую загрузку во время выполнения. Оба подхода имеют свои преимущества и области применения, что позволяет проводить сравнительный анализ их характеристик.