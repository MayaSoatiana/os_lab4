\section{Выводы}

В ходе выполнения лабораторной работы были успешно достигнуты поставленные цели:

\begin{enumerate}
    \item \textbf{Освоены принципы создания динамических библиотек:} На практике реализованы две динамические библиотеки (DLL) с различными алгоритмами выполнения одинаковых контрактов. Изучены особенности экспорта функций через макросы \texttt{\_\_declspec(dllexport)} и использование \texttt{extern "C"} для предотвращения декорирования имен в C++

    \item \textbf{Освоено статическое использование библиотек:} Разработана программа №1, которая использует библиотеку на этапе компиляции. На практике применен процесс статической линковки, когда код библиотеки включается в исполняемый файл. Изучены преимущества данного подхода: высокая производительность и отсутствие зависимостей во время выполнения

    \item \textbf{Освоено динамическое использование библиотек:} Создана программа №2, которая загружает библиотеки во время выполнения через Windows API (\texttt{LoadLibraryW()}, \texttt{GetProcAddress()}, \texttt{FreeLibrary()}). Реализована возможность переключения между различными реализациями функций без перекомпиляции программы

    \item \textbf{Реализована обработка системных ошибок:}
    \begin{itemize}
        \item Обработка ошибок загрузки динамических библиотек
        \item Контроль доступности экспортируемых функций
        \item Валидация входных данных для математических функций
        \item Корректное освобождение ресурсов при завершении работы
    \end{itemize}
\end{enumerate}

\subsection{Сравнительный анализ методов использования библиотек}

Работа продемонстрировала принципиальные различия между двумя подходами к использованию библиотек:

\begin{itemize}
    \item \textbf{Статическая линковка (программа №1):}
    \begin{itemize}
        \item Преимущества: высокая производительность, простота развертывания (один исполняемый файл), надежность
        \item Недостатки: невозможность обновления библиотеки без перекомпиляции, больший размер исполняемого файла
        \item Применение: когда требуется максимальная производительность и стабильность
    \end{itemize}

    \item \textbf{Динамическая загрузка (программа №2):}
    \begin{itemize}
        \item Преимущества: гибкость (возможность переключения реализаций), меньший размер исполняемого файла, возможность обновления без перекомпиляции
        \item Недостатки: дополнительные накладные расходы на загрузку, зависимость от наличия DLL файлов, сложность отладки
        \item Применение: плагинная архитектура, системы с модульной структурой, когда требуется гибкость конфигурации
    \end{itemize}
\end{itemize}

\subsection{Алгоритмические аспекты}

В ходе работы были реализованы и протестированы различные алгоритмы:
\begin{itemize}
    \item \textbf{Наивный алгоритм проверки простоты чисел:} Прост в реализации, но имеет сложность O(n²), что делает его неэффективным для больших диапазонов
    \item \textbf{Алгоритм "Решето Эратосфена":} Более сложная реализация, но имеет сложность O(n log log n), что значительно эффективнее для больших диапазонов
    \item \textbf{Геометрические вычисления:} Демонстрация полиморфизма через разные формулы для вычисления площади
\end{itemize}

\subsection{Практическая ценность}

Полученный опыт может быть применен при разработке:
\begin{itemize}
    \item Модульных приложений с плагинной архитектурой
    \item Систем, требующих возможности "горячей" замены компонентов
    \item Кроссплатформенного ПО с поддержкой разных алгоритмов
    \item Библиотек для повторного использования кода
\end{itemize}

Лабораторная работа успешно продемонстрировала современные подходы к модульному программированию и показала важность правильного выбора архитектуры в зависимости от требований проекта.

\pagebreak