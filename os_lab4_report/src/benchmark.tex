\section{Результаты}

В результате работы была разработана система динамических библиотек и две тестовые программы, демонстрирующие разные подходы к использованию библиотек. Программа успешно функционирует в операционной системе Windows и может быть адаптирована для Unix-систем.

\subsection{Ключевые особенности реализации}

\begin{itemize}
    \item \textbf{Две динамические библиотеки:} Созданы библиотеки mathlib1.dll и mathlib2.dll с различными реализациями одинаковых контрактов
    \item \textbf{Корректная экспорт/импорт функций:} Использование \texttt{\_\_declspec(dllexport)} и \texttt{extern "C"} для обеспечения совместимости
    \item \textbf{Два метода использования библиотек:}
    \begin{itemize}
        \item Статическая линковка на этапе компиляции (программа №1)
        \item Динамическая загрузка во время выполнения (программа №2)
    \end{itemize}
    \item \textbf{Единый интерфейс пользователя:} Обе программы поддерживают одинаковый набор команд (0, 1, 2, help, quit)
    \item \textbf{Переключение реализаций:} Программа №2 позволяет переключаться между библиотеками без перекомпиляции
    \item \textbf{Обработка ошибок:} Проверка корректности входных данных, загрузки библиотек и доступности функций.
\end{itemize}

\subsection{Пример работы программы}

\begin{verbatim}
PS D:\MAI-year2\os\os_lab4\build> bin/program1.exe

=== Program 1: Static Linking ===
Library linked at COMPILE TIME
Commands:
  0 - Show info about current implementation
  1 A B - Count primes between A and B
  2 A B - Calculate area with sides A and B
  help - Show this help
  quit - Exit program

> 1 35 92
PrimeCount(35, 92) = 13

> 2 5 16
Square(5, 16) = 80

> quit
Goodbye!
PS D:\MAI-year2\os\os_lab4\build> bin/program2.exe
Successfully loaded Library 1 by default

=== Program 2: Dynamic Loading ===
Libraries loaded at RUNTIME
Current library: Library 1
Commands:
  0 - Switch between Library 1 and Library 2
  1 A B - Count primes between A and B
  2 A B - Calculate area with sides A and B
  help - Show this help
  quit - Exit program

[Lib1] > 1 24 89
PrimeCount(24, 89) = 15

[Lib1] > 2 4 20
Square(4, 20) = 80

[Lib1] > 0
Switched to Library 2
  PrimeCount: Sieve of Eratosthenes
  Square: Right triangle area (A × B / 2)

[Lib2] > 1 24 89
PrimeCount(24, 89) = 15

[Lib2] > 2 4 20
Square(4, 20) = 40

[Lib2] > quit
Goodbye!
\end{verbatim}

\subsection{Производительность}

\begin{itemize}
    \item \textbf{Время загрузки:} Программа №1 запускается быстрее, так как библиотека уже залинкована
    \item \textbf{Время выполнения функций:} Одинаково для обоих методов после загрузки библиотеки
    \item \textbf{Потребление памяти:}
    \begin{itemize}
        \item Программа №1: больше памяти (код библиотеки в процессе)
        \item Программа №2: меньше памяти (библиотека загружается при необходимости)
    \end{itemize}
    \item \textbf{Алгоритмическая эффективность:}
    \begin{itemize}
        \item Библиотека 1: O(n²) для PrimeCount, подходит для небольших n
        \item Библиотека 2: O(n log log n) для PrimeCount, эффективна для больших n
    \end{itemize}
\end{itemize}